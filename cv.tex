\documentclass[11pt,a4paper,sans]{moderncv}
\moderncvstyle{classic}
\moderncvcolor{black}

\usepackage[utf8]{inputenc}
\usepackage[export]{adjustbox}
% adjust the page margins
\usepackage[scale=0.75]{geometry}

\name{Markus}{Pettersson}
\title{Computer Science Student}                               
\address{Luftvärnsvägen 2}{415 27 Göteborg}{Sweden}
\phone[fixed]{+46~725~207~118}
\email{markus.pettersson1998@gmail.com}
\homepage{datamaskin.se}
\social[github]{MarkusPettersson98}
\social[linkedin][www.linkedin.com/in/markus-pettersson-4785a1188/]{markus-pettersson-4785a1188}

% Be gone, compilation errors
\recipient{\mbox{}}{}
%----------------------------------------------------------------------------------
\begin{document}

\makecvtitle

% Relevant Experience(s)
\section{Experience}
\cventry{2022--present}{IT Consultant}{Agreat}{}{Gothenburg}{IT Consultant specialized in CI/CD automation and DevOps practices, with a sprinkle of Agile. I help teams enable their software developers to take complete ownership of their services.}

\cventry{Oct 2022--\\present}{DevOps Engineer}{Agreat}{Volvo Car Corporation}{Gothenburg}{As one of two members on the platform team at Volvo Cars' Digital Key solution, I enable both embedded engineers and cloud engineers alike to focus on their deliveries by automating repetetive and error prone tasks. During a single day I can go from soldering new cable harnesses needed for debugging CAN-busses to implementing and deploying Jenkins CI/CD pipelines, depending on what's needed. Not entirely unlike a technologically versed janitor.}
\cvitemwithcomment{}{\underline{Programming languages}: Python, bash, Embedded C/C++ (Arduino)}{}
\cvitemwithcomment{}{\underline{Other}: Jenkins}{}

\cventry{Sep 2022--\\Oct 2022}{DevOps \& Software Architect}{Agreat}{Hypocampus}{Gothenburg}{In close collaboration with \href{https://www.hypocampus.se/}{Hypocampus}, a digital learning platform for medical students, a tool for importing PDF-based exams to Hypocampus' interactive practice sessions was developed. I took responsibility for setting up and maintaining the CI infrastructure and code practices, which allowed the team of 5 new grads to focus on tackling the problem while keeping a rapid pace of development without fear of breaking things. First assignment during start up phase at Agreat.}
\cvitemwithcomment{}{\underline{Programming languages}: Python}{}
\cvitemwithcomment{}{\underline{Other}: Gitlab CI, Docker, mypy, pytest}{}

\cventry{2021}{Backend Web Developer}{Tele Radio}{}{Gothenburg}{Summer internship as a backend developer within the Business Software team at Tele Radio. During my stay, two new internal applications with a focus on automation and standardization of accounting and document management were developed using the Django web framework. These services were successfully deployed to Tele Radio's intranet. I also contributed to some old services through various patches.}
\cvitemwithcomment{}{\underline{Programming languages}: Python (Django), Javascript}{}
\cvitemwithcomment{}{\underline{Other services used}: AWS (ECS, S3), Docker}{}

\cventry{2020--present}{Report Management System for truck drivers}{Markus Pettersson}{}{Gothenburg}{A mobile application with focus on ease for truck drivers was designed, developed and delivered through my then newly founded company. In collaboration with the owner of a small-sized truck driving company, we identified the need for a better Report Management System for reporting common errors which drivers could face during their day-to-day work. The app has been in use company-wide since it was launched. I taught myself how to launch an application on the Android and iOS platforms, as well as how to develop and interface with platform native code from a React Native application.}
\cvitemwithcomment{}{\underline{Programming languages}: Javascript (React Native, Redux), Kotlin, Swift.}{}
\cvitemwithcomment{}{\underline{Other services used}: Firebase, Google Play, App Store.}{}

\section{Education}
\cventry{2022--2022}{Master Thesis}{Chalmers University of Technology}{Gothenburg}{}{{}
  \href{https://github.com/dpella/masterthesisproposals/blob/main/2022/intervals_for_sensitivity.md}{Sensitivity computation for user-defined functions in Differential Privacy systems.}
  We used Haskell's advanced type system and type level programming to implement a minimal framework for doing efficient range analysis on linear queries.
  We benchmarked our solution by re-implementing \href{https://en.wikipedia.org/wiki/Exponential_mechanism_(differential_privacy)}{the exponential mechanism} in \href{https://www.dpella.io/}{DPella}'s database solution, which in turn offers \href{https://en.wikipedia.org/wiki/Differential_privacy}{differential privacy} guarantees.
}
\cventry{2020--present}{M.Sc. Computer Science - Algorithms, languages and logic}{Chalmers University of Technology}{Gothenburg}{}{{}Computer Science is a master degree program with the main focus on the foundations in the science of programming. I am focusing my education towards compilers and programming language design, which heavily emphasises logic, reasoning and functional programming. }
\cventry{Spring 2020}{Bachelor Thesis}{Chalmers University of Technology}{Gothenburg}{}{{}
Dynamically Recompiling Emulator.
The thesis examined the possibility to retarget machine code compiled for a classic 8-bit processor (MOS 6502) to modern systems by leveraging libraries for compilation and optimizations from the LLVM compiler framework. A JIT-compiler was successfully developed and presented in May 2020.}
\cventry{2017--2020}{B.Sc Software Engineering}{Chalmers University of Technology}{Gothenburg}{}{{}}

\section{Certificates}
\cventry{2022}{Scrum Master Certification}{Scrum Alliance}{}{Gothenburg}{2 day course in Scrum, agile product development and group dynamics. Course was held by \href{https://www.crisp.se/kurser/kurstyper/certified-scrummaster}{Crisp}. Certificant ID \textbf{001412398} issued by \href{https://www.scrumalliance.org/get-certified/scrum-master-track/certified-scrummaster}{Scrum Alliance}.}

\section{Skills}
\cvitemwithcomment{Languages}{Swedish (Native), English (Fluent)}{}

\cvitemwithcomment{Progamming Languages}{Java, Javascript, Python, Haskell, Rust}{}
\cvitemwithcomment{Frameworks}{React, React Native, Django, Javalin}{}
\cvitemwithcomment{Databases}{PostgreSQL, Firebase, Redis}{}
\cvitemwithcomment{Other tools}{git, Linux, Bash, Make, Docker, Github, Github Actions, Gitlab CI, LLVM, \LaTeX.}{}

\section{Extracurricular activities}
\cvlistitem{\textbf{Rekryt D/IT}\\ Rekryt D/IT is a society at Chalmers with the aim of reaching out to high school students and guide them in their choice of university programme. I worked with answering individual high school student's questions via online message correspondence throughout the school year 2020/2021. I also worked with arranging the booth for the programmes Data and Informationsteknologi at Chalmersdagen as well as multiple online-open house sessions for those interested in the programmes.}

\cvlistitem{\textbf{DatE-IT}\\ DatE-IT is a labor fair for students of computer engineering, electrical engineering, software engineering, and medical engineering at Chalmers. I assisted with logistics during the construction of the show floor for the fair of 2021. I was the main contact person for the people representing Spark Vision, which included answering questions and making sure they got the most out of their time at the fair.}

\cvlistitem{\textbf{Driver's License}\\ Yes.}

\end{document}
